\section{Software Package}


\begin{figure}[h]
\centering
\includegraphics[width =80mm]{13.jpg}
\includegraphics[width = 70mm]{14.jpg}
\end{figure}




\textbf{LabTracer} is a program for characterizing components using SourceMeters and Series SourceMeters in a research environment. The program makes it easy to build tests for the Keithley Series 2600, Series 2400 and Model 6430 SourceMeters using a GUI interface.
\newline \textbf{Origin} is a proprietary computer program for interactive scientific graphing and data analysis. It is produced by OriginLab Corporation, and runs on Microsoft Windows. Origin is an easy-to-use data analysis and graphing software and is used in statistics, signal processing, curve fitting and peak analysis. Version 8.0 was used for this experiment.

\begin{figure}[h]
\centering
\includegraphics[width =40mm]{15.jpg}
\includegraphics[width = 60mm]{16.jpg}
\end{figure}


\section{I-V Characterstics}


The chief property of photovoltaic device (PV) is the Current-Voltage curve or the I-V characteristic which measures all the possible combinations of current and voltage output. As mentioned above, the maximum current occurs when there is no resistance (Short Circuit Current) while the maximum voltage occurs when there is a break in the circuit (Open Circuit Voltage). These curves are generally used to define basic parameters of the component. The I-V reading of the devices are as follows:

\begin{figure}[h]
\centering
\includegraphics[width = 100mm]{17.jpg}
\caption{I-V Characterstics :Light}
\label{figure:8}
\end{figure}

\begin{figure}[h]
\centering
\includegraphics[width = 100mm]{18.jpg}
\caption{I-V Characterstics :Dark}
\label{figure:9}
\end{figure}

\pagebreak

\begin{figure}[h]
\centering
\includegraphics[width = 100mm]{19.jpg}
\caption{Plot : I-V Characterstics-Light  }
\label{figure:10}
\end{figure}

\begin{figure}[h]
\centering
\includegraphics[width = 100mm]{20.jpg}
\caption{Plot : I-V Characterstics-Dark }
\label{figure:11}
\end{figure}

\pagebreak

\begin{figure}[h]
\centering
\includegraphics[width = 100mm]{21.jpg}
\caption{I-V Characterstics :Device 2 }
\label{figure:12}
\end{figure}

\begin{figure}[h]
\centering
\includegraphics[width = 100mm]{22.jpg}
\caption{I-V Characterstics :Device 3}
\label{figure:13}
\end{figure}

\pagebreak]

\begin{figure}[h]
\centering
\includegraphics[width = 100mm]{23.jpg}
\caption{Plot : I-V Characterstics-Device 2  }
\label{figure:14}
\end{figure}

\begin{figure}[h]
\centering
\includegraphics[width = 100mm]{24.jpg}
\caption{Plot : I-V Characterstics-Device 3 }
\label{figure:15}
\end{figure}






\clearpage

\section{Analysis}
The result obtained are compiled as given above.The given parameters were calculated on the basis of theoretical standards and constants taken as per recommendation and various journals. The need for taking various parameters on the basis of light intensity was also considered and the devices are observed on the basis of different light conditions (Device A for Bright Light, Device B under dark conditions). The results observed were substandard and were however under the theoretical values. The current observed was in the lower range likely due to presence of impurities. The other parameters however were acceptable and as expected. There were a few discrepancies in the calculations and readings which was resolved by taking suitable adjustments in the final values. Overall the experiment to synthesize a prototype device was successful and worked under proper conditions. The values however recorded were substandard and this can be attributed to various factors such as proper deposition, impurities, electrolyte solution. 
\begin {table}[t]
\begin{center}
 \begin{tabular}{||c | c | c | c | c | c||} 
 \hline
 Device & Short Circuit Current & Open Circuit Voltage & Maximum voltage & Maximum 
Current & Fill Factor
  \\ [0.5ex] 
 \hline\hline
 A & 0.6016 & -0.001898 & 0.564 & -0.0015 & 0.78 \\ 
 \hline
 B & 0.1638 & -0.000232 & 0.113 & -0.0001 & 0.337\\
 \hline
 C & 0.5388 & -0.0002621 & 0.465 & -0.000196 & 0.648\\
 \hline
 
\end{tabular}
\caption{Experimental Results}
\end{center}
\end {table}

 
