
 One of the most important source of regenerative energy and perhaps representing mankind’s only inexhaustible energy source (and also the source of waterpower, wind and biomass) is Solar light. The annual energy input of solar irradiation on earth exceeds by thousand times the worlds yearly energy consumption (43% visible, 52% IR, 5% UV). However for this development, fundamentally new technology is important. This is where photovoltaic cells come up as it is one of today’s most promising tools to make use of solar energy for its direct conversion.
\section{Energy}
Energy has a large number of different forms, and there is a formula for each one. These are: gravitational energy, kinetic energy, heat energy, elastic energy, electrical energy, chemical energy, radiant energy, nuclear energy, mass energy. If we total up the formulas for each of these contributions, it will not change except for energy going in and out. It is important to realise that in physics today, we have no knowledge of what energy is. We do not have a picture that energy comes in little blobs of a definite amount. It is not that way. However, there are formulas for calculating some numerical quantity. [1]

Energy is usually measured in the unit of Joule (J), named after the English physicist James Prescott Joule (1818-1889), which it defined as the amount of energy required applying the force of 1 Newton through the distance of 1 m, 1 J = 1 Nm. 1 J is a very small amount of energy compared to the human energy consumption. Therefore, in the energy markets, such as the electricity market, often the unit Kilowatt hour (kWh) is used. On the other hand, the amounts of energy in solid state physics, the branch of physics that we will use to explain how solar cells work, are very small. Therefore we will use the unit of electron volt, which is the energy a body with a charge of one elementary charge (e = 1.602 × 10−19 C) gains or losses when it is moved across a electric potential difference of 1 Volt (V), 
1 eV = e × 1 V = 1.602 × 10−19 J.

\section{Initiative to Meet Energy Crisis}
\begin{figure}[h]
\centering
\includegraphics[width = 80mm]{2.jpg}
\caption{World Energy Supply }
\label{figure:1}
\end{figure}
Today's mix of energy use consists of 36 percent oil, 24 percent gas, 28 percent coal, six percent nuclear, six percent hydroelectric and one percent renewables, like wind and solar energy.
Planet Earth is facing an energy crisis owing to an escalation in global energy demand, continued dependence on fossil-based fuels for energy generation and transportation, and an increase in world population, exceeding seven billion people and rising steadily. Excessive burning of fossil fuels is not only depleting natural resources, but is resulting in a steady increase of carbon dioxide emissions, which experts believe is responsible for increasing average global temperatures. While natural cyclical variations do occur in regional and global climates, there is now widespread agreement among scientific communities and governments that recent climate change is accelerating as a result of human intervention and that rapid and profound measures will be required to reduce harmful impacts.

\begin{figure}[h]
\centering
\includegraphics[width = 90mm]{3.jpg}
\caption{Total Energy Consumption }
\label{figure:2}
\end{figure}

According to the International Energy Agency (IEA), India needs to in the next twenty years. Even though India possesses a rich heterogeneous mix of energy components, ambitious but unclear policies have created a difficult environment for potential investors. Thus, it will require extraordinary effort, innovative vision and viable solutions to meet the increasing demand for energy while maintaining an eco-friendly approach. [2]
Gas, oil, coal and renewable energy are the main energy sources of India. However, certain pertinent questions still arise.\cite{saini2010alternative}

\section{Solutions and Cost Factor}
Fortunately, there are two key ways we can maximize energy while offsetting the cost of it, and that is by 1) Using more alternative energy sources, 2) Making the things that use energy intelligent

The report aims to present the sun as the only energy source that can meet the oil depletion challenge. But solar energy ramp-up must be large-scale and immediate. Some of the factors that we can consider are
\newline \textbf {Cost} – Solar Power has always been viewed as an expensive technology but competition in the solar market in recent years has resulted in a drastic fall in installation prices. A solar PV system installed today costs less than half of what it did two years ago and it is likely that this cost could fall even further as we develop more affordable technologies for capturing energy from the sun.
\newline \textbf {Efficiency and space} – renewable energy critics will often complain of the low efficiency rates of solar PV panels, which are currently around 15, but when your fuel source is drawn from an infinite supply then this is only really an issue in terms of space. The great thing about solar Power is that systems are flexible and easily scalable. 
\newline \textbf {Intermittency and storage} – like most renewables, solar PV’s biggest flaw is intermittency i.e. an irregular supply of electricity because the panels only produce a yield when the sun is shining. If we are to seriously invest in a solar electricity supply then we must first overcome this challenge with suitable storage technology.\cite{imre2006majority}.

   