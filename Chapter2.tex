\section{Introduction}
A solar cell, or photovoltaic cell (previously termed "solar battery"), is an electrical device that converts the energy of light directly into electricity by the photovoltaic effect. Most photo-voltaic solar panels are silicon based or a variation of. There are several different types of solar panel including tiles, film, and lightweight. Amorphous Solar Cells, Bio hybrid Solar Cell , Dye-Sensitized Solar Cell (DSSC) , Hybrid Solar Cell. Our main focus here will be Dye Sensitized Solar cell. Dye Sensitized solar cells (DSSC), also sometimes referred to as dye sensitized cells (DSC), are a third generation solar (photovoltaic) cell that is used to convert visible light into electrical energy. They were invented in 1991 by the researchers in École Polytechnique Fédérale de Lausanne , Switzerland. One of the main advantages of this over traditional solar cell is its many features such as simple to make, semi-flexible, semi-transparent and low cost material. DSSCs are currently the most efficient third-generation and this makes DSSCs attractive as a replacement for existing technologies in "low density" applications like rooftop solar collectors. . They have a good efficiency (about 10-14 percent) even under low flux of sunlight. However, a major drawback is the temperature sensitivity of the liquid electrolyte. Hence a lot of research is going on to improve the electrolyte’s performance and cell stability.
\begin{figure}[h]
\centering
\includegraphics[width = 70mm]{4.jpg}
\includegraphics[width = 60mm]{5.jpg}
\end{figure}


\section{Structure and Principle}
\begin{figure}[h]
\centering
\includegraphics[width = 100mm]{6.jpg}
\caption{DSSC Structure}
\label{figure:3}
\end{figure}

The anode of a DSC consists of a glass plate which is coated with a transparent conductive oxide (TCO) film. Indium tin oxide (ITO) or fluorine doped tin oxide are most widely used. A thin layer of titanium dioxide (TiO2) is applied on the film. The semiconductor exhibits a high surface area because of its high porosity.
\newline The anode is soaked with a dye solution which bonds to the TiO2. The dye – also called photosensitizers – is mostly a ruthenium complex or various organic metal free compounds. For demonstration purposes, also plain fruit juice (such as from blackberries or pomegranates) can be used. They contain pigments which are also able to convert light energy into electrical energy.
\newline 					The cathode of a DSC is a glass plate with a thin Pt film which serves as a catalyst. An iodide/triiodide solution is used as the electrolyte. Both electrodes are pressed together and sealed so that the cell does not leak. An external load can be powered when light shines on the anode of the dye solar cell. The dye is the photoactive material of DSSC, and can produce electricity once it is sensitized by light. [4]




\begin{figure}[h]
\centering
\includegraphics[width = 100mm]{7.jpg}
\caption{Principle of DSSC  }
\label{figure:4}
\end{figure}
\textbf{STEP 1:}The dye molecule is initially in its ground state (S). The semiconductor material of the anode is at this energy level (near the valence band) non conductive.
When light shines on the cell, dye molecules get excited from their ground state to a higher energy state (S*), 
\newline \textbf{STEP 2:}  The excited dye molecule (S*) is oxidized (see equation 2) and an electron is injected into the conduction band of the semiconductor.  Electrons can now move freely as the semiconductor is conductive at this energy level.
\newline \textbf{STEP 3:}  The oxidized dye molecule (S+) is again regenerated by electron donation from the iodide in the electrolyte
\newline \textbf{STEP 4:}  In return, iodide is regenerated by reduction of triiodide on the cathode
\newline It is the movement of these electrons that creates energy which can be harvested into a rechargeable battery, super capacitor or another electrical device.


\section{Efficiency and Other Parameters}

The efficiency η is the ratio between maximum generated power Pmax and electrical input power Pin from the light source.
\begin{center}
{\psfig{figure= Figures/8.jpg ,width=40mm }\\}
\end{center}
Dye sensitized solar cell have a theoretical maximum energy conversion efficiency of 33 percent; however due to technical constraints, the actual energy conversion efficiency of DSSC is closer to 11 percent, which is less than half of the crystalline silicon based solar cells efficiency of 24.4 percent .Improving DSSC efficiency is critical to widespread adoption of this technology.


\vspace{5mm}
\textbf{Schematic diagram of I-V curves with and without light}
\begin{figure}[h]
\centering
\includegraphics[width = 100mm]{9.jpg}
\caption{I-V characteristics }
\label{figure:5}
\end{figure}

\textbf{Current flux} is nearly constant at lower potentials.  It reaches its maximum when the potential is zero.  The generated current decreases with increasing potential.  It is zero at the open circuit potential.  The cell can get damaged at excessively high values. The short circuit current ISC is the highest current that can be drawn from a solar cell. The short circuit current increases with increasing light intensity. Meanwhile, The open circuit potential EOC is the highest voltage of a solar cell at a given light intensity. It is also the potential where current flow through a solar cell is zero. The Fill factor (FF) is an important parameter to specify the overall capabilities of a cell.  It describes the quality and idealness of a solar cell.
\begin{figure}[h]
\centering
\includegraphics[width = 100mm]{11.jpg}
\caption{Fill Factor Representation }
\label{figure:6}
\end{figure}

\textbf{The Fill factor} is the ratio of maximum generated power Pmax to theoretical power maximum Ptheo of a solar cell. The general formula for the Fill Factor is:
\begin{center}
{\psfig{figure= Figures/10.jpg ,width=40mm }\\}
\end{center}
EMP and IMP are potential and current of the I-V curve where the generated power is at the maximum.




\section{Electrophoretic Deposition}
\begin{figure}[h]
\centering
\includegraphics[width = 100mm]{12.jpg}
\caption{EPD Deposition }
\label{figure:7}
\end{figure}

Electrophoretic deposition EPD is a method of coating a conductive part with particles suspended in a fluid dispersion under the influence of an electric field applied between the work part and the counter electrode. Similar to Electroplating coating electrophoretic deposition utilizes electrically charged particles moving between two electrodes (an anode and a cathode) immersed in a liquid media.However in contrast to conductive electrolytes used in electroplating, the fluids of electrophoretic dispersions are dielectric.
It involves a broad range of industrial processes which includes electrocoating, cathodic electrodeposition, anodic electrodeposition, and electrophoretic coating, or electrophoretic painting. The process is useful for applying materials to any electrically conductive surface. The materials which are being deposited are the major determining factor in the actual processing conditions and equipment which may be used.

\vspace{4mm}
\textbf{Implementation}
\newline 1. Place two electrodes into a mixture of solvent and particles.
\newline 2. Apply a voltage/current between the electrodes for a set period of time.
\newline 3. If after removal one of the electrodes has a coating of particles, then EPD has occurred.
