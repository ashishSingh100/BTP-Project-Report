A demonstration and experiment was done initially to understand the basics of experimentation and the resulting effect it has on the values and calculation. The initial stage was cleaning of the glass substrate. The Whole experiment was done under the supervision of our project guide in the Material Synthesis laboratory and it�s adjoining Advanced Materials Laboratory.
\section{Cleaning}
\begin{figure}[h]
\centering
\includegraphics[width = 80mm]{25.jpg}
\end{figure}
\textbf{Glass Substrate} is a thin glass board on which a thin circuit is deposited with precision. It is a key material for LCD. They are the individual plies of glass used to fabricate glass units and may also be referred to as float glass, raw glass or glass lites. Glass substrate options include clear, tinted and low iron.
\newline \textbf{Sonication}
\newline Sonication is a process in which sound waves are used to agitate particles in solution.  Such disruptions can be used to mix solutions, speed the dissolution of a solid into a liquid (like sugar into water), and remove dissolved gas from liquids.  Sonicator Bath has been designed using MOSFET/IGBT and latest Microcontroller base technology. [6]


\begin{figure}[h]
\centering
\includegraphics[width = 150mm]{26.jpg}
\end{figure}\textbf{Components Involved and Steps followed}
\newline1) Ethanol
\newline2) Isopropyl Alcohol
\newline3) Distilled Water (Soap Solution)
\newline4) Acetone
\newline Firstly, clean the beakers with ethanol solution. The beaker was then dispersed in Soap solution and the glass slides to be cleaned were further put inside and then it was washed properly. Another beaker filled with distilled water was taken and the glass slides were put and then the resulting solution was cleaned and sonicated with digital ultrasonic cleaner for 2 minutes.
\clearpage
\begin{figure}[h]
\centering
\includegraphics[width = 70mm]{27.jpg}
\includegraphics[width = 85mm]{28.jpg}
\end{figure}

The slides were then put into \textbf{ethanol} for 2 minutes and then the same cleaning procedure was repeated. Isopropyl Alcohol was used in the next step and the glass slides were sonicated further along with the solution. The same glass slides were then washed and cleaned with acetone. 
The main reasons for cleaning in the specific order was that the \textbf{Acetone} will remove organic impurities from the chip substrates, and is particularly apt for dissolving oily or greasy contaminants.  However, acetone evaporates rapidly and will redeposit the contaminants.  Methanol works very well to dissolve the acetone with its contaminants, without rapidly evaporating. \textbf{Isopropanol} (a.k.a. isopropyl alcohol) is an excellent rinse agent for the methanol and contaminated acetone, and for lifting and removing particles. 
\begin{figure}[h]
\centering
\includegraphics[width = 70mm]{29.jpg}
\includegraphics[width = 70mm]{31.jpg}
\end{figure}
 
\vspace{30 mm} 
The net cleaning effect was clearly visible with proper current values and no specks being present.

\vspace{4 mm}

\begin{figure}[h]
\centering
\includegraphics[width = 110mm]{30.jpg}
\end{figure}

\section{Layer Deposition}

\subsection{\ce{TiCl4} treatment}
\begin{figure}[h]
\centering
\includegraphics[width = 90mm]{32.jpg}
\end{figure}

Titanium tetrachloride is a titanium and chlorine compound and has the chemical formula \ce{TiCl4}, tetra coming from the four chlorine atoms. It is also known as Tetrachlorotitanium or Titanic chloride. It appears as a dense, colorless to pale yellow distillable liquid. It generally works as complement to the titanium dioxide deposition that is soon followed with it. \ce{TiCl4} increases the electron conductance on the interface of the FTO glass. A suitable amount of \ce{TiCl4} in the film could also provide a large surface area for dye adsorption
\vspace{4 mm}
\begin{figure}[h]
\centering
\includegraphics[width = 70mm]{33.jpg}
\includegraphics[width = 70mm]{34.jpg}
\end{figure}

Titanium dioxide (\ce{TiO2}) electrodes are vital components for the fabrication of dye-sensitized solar cells. Titanium tetrachloride (\ce{TiCl4}) treatment is usually adopted as a pre- or post-treatment for the improvement of \ce{TiO2} electrodes in DSSCs. Previous journals have associated these treatments with the improvement of bonding strength between the fluorinated tin oxide (FTO) substrate and the porous \ce{TiO2} layer. Recent studies and reports have also pointed in a healthy increase of the overall conversion efficiency using this deposition. [8]
\begin{figure}[h]
\centering
\includegraphics[width = 75mm]{35.jpg}
\includegraphics[width = 75mm]{36.jpg}
\end{figure}

\textbf{Calcination} is the process of subjecting a substance to the action of heat, but without fusion, for the purpose of causing some change in its physical or chemical constitution. A Hot air oven is used to condition the specimens to the specified temperature in the laboratory. The final samples are then ready for TiO2 treatment.
\begin{figure}[h]
\centering
\includegraphics[width = 120mm]{37.jpg}
\end{figure}
 
\subsection{\ce{TiO2} treatment}

Titanium dioxide layers on glass substrates have been widely studied for optical and electronic applications because they have a 
\begin{itemize}
\item High refractive index (up to 2.7), 
\item High Photo catalytical activity 
\item Good physical and chemical Stability
\item Wide Band Gap (greater than 3 eV)
\item High Dielectric Constant
\item Good Optical Properties
\end{itemize}

\begin{figure}[h]
\centering
\includegraphics[width = 100mm]{38.jpg}
\caption{\ce{TiO2} Structure}
\label{figure:16}
\end{figure}
\vspace{4mm}
The metal oxide should provide mesoporous structure for dye adsorption and the bandgap edge levels should be adjusted with respect to the Dyes and electrolytes. Other metal oxides can also be used like ZnO and SnO2. Several research groups have tried and concluded that their efficiency is low compared to TiO2, because of less mesoporous structure in ZnO and SnO2 etc.Internal network structure is important to achieve high charge collection efficiency and more electron transportation. TiO2 is more stable compare to other metal oxides that is why it is the preferred choice.
\newpage
\textbf{Components Involved and steps followed}    
\newline 1) Acetyl Acetone (32 ml)
\newline 2) Titanium Dioxide (0.32 gm)
\newline 3) Iodine Powder (0.032 gm)

The ingredients are put together and the Resulting solution is mixed after carefully measuring the values. The Solution is further sonicated for 30 
Minutes. It is then stirred for 20-30 Minutes at room temperature using Magnetic Stirrer

\begin{figure}[h]
\centering
\includegraphics[width = 70mm]{39.jpg}
\includegraphics[width = 70mm]{41.jpg}
\end{figure}

Connect the Power supply Clamp to our glass samples opposite to a reference counter electrode. Dip the Clamp with the Samples in the stirred solution. Turn on the power for the regulated supply and let it run for approx. 80-100 seconds. (Under the proper specification)

\textbf{Power Supply}: Voltage: 20 V
\newline \hspace*{30mm}     Current Amplitude: 100 mA

\textbf{Magnetic Stirrer}: Voltage- 220/230 VAC
\newline \hspace*{30mm} Temperature Range-250 Degree Celsius
			  
			  
\begin{figure}[h]
\centering
\includegraphics[width = 140mm]{42.jpg}
\end{figure}

The experimental setup is as shown above.

\begin{figure}[h]
\centering
\includegraphics[width = 100mm]{43.jpg}
\end{figure}

Remove the Layer deposited chip carefully and lightly clean away the outer edges. The Results of the TiO2 film deposition are as given. We can notice a clear white thin deposition on the glass chips. These Substrates are then deposited in a Microwave furnace/Oven for Calcination.

\subsection{Dye Solution}

\textbf{Composition}
\begin{itemize}
\item Dye particles (0.0036 gm.)
\item Ethanol Solution (10 ml)
\end{itemize}

Example of dye molecule that is used in DSSC:
\begin{itemize}
\item Ttriscarboxy-rutheniumterpyridine [Ru(4,4,4"-(COOH)3-terpy)(NCS)3] 
(Black dye) 

\item 1-ethyl-3 methylimidazolium tetrocyanoborate [EMIB(CN)4]thanol Solution (10 ml)
\item Copper-diselenium [Cu(In,GA)Se2]
\end{itemize}

\textbf{MK Dye} material was used however for this experiment.

\vspace{5mm}

After preparing the dye solution, we place the glass samples into Dye sol. for 24 hours under dark conditions. This is done due to the consideration of the samples being light sensitive under room conditions. After a period of 24 hours, the samples are removed and processed further. Some dye impurities are further removed by using ethanol with dropper on glass surface.

\newpage

\begin{figure}[h]
\centering
\includegraphics[width = 100mm]{44.jpg}
\end{figure}

The final Samples are visible in the rightmost region. We can observe a clear coating of dark maroon color across the glass slides. 

\begin{figure}[h]
\centering
\includegraphics[width = 45mm]{45.jpg}
\includegraphics[width = 45mm]{46.jpg}
\end{figure}
A before and after change is clearly seen in the samples.

\newpage
\begin{figure}[h]
\centering
\includegraphics[width = 140mm]{47.jpg}
\caption{Plot : Dye Layer }
\label{figure:16}
\end{figure}
\textbf{Importance}

The dye injects an electron into the conduction band of the TiO2 layer. The dyes in dye-sensitized solar cells (DSSCs) require one or more chemical substituents that can act as an anchor, enabling their adsorption onto a metal oxide substrate. This adsorption provides a means for electron injection, which is the process that initiates the electrical circuit in a DSSC. The dye molecules are quite small (nanometer sized), so in order to capture a reasonable amount of the incoming light the layer of dye molecules needs to be made fairly thick, much thicker than the molecules themselves.[7]


\newpage
\section{Negative Electrode and Electrolyte Solution}

\begin{figure}[h]
\centering
\includegraphics[width = 80mm]{48.jpg}
\includegraphics[width = 70mm]{49.jpg}
\end{figure}

Cleaned FTO samples are treated with Platinum solution. Further Calcination at 450 degree Celsius is done. The Negative electrode (platinum based) is prepared using this method.

\vspace{4mm}
\textbf{Electrolyte Solution}
\begin{figure}[h]
\centering
\includegraphics[width = 120mm]{50.jpg}
\end{figure}
\newline The components involved in Electrolyte solution was depicted above.

\newpage


\begin {table}[t]
\begin{center}
 \begin{tabular}{||c | c ||} 
 \hline
 Solution & Quantity
  \\ [0.8ex] 
 \hline\hline
 LiI (Lithium Iodide) & 0.0669 gm \\ 
 \hline
I2 (Iodine) & 0.0634 gm. \\
 \hline
 TBP (4-tert. Butylpyridine ) & 0.3380 gm \\
 \hline
  Acetonitrile & 5 ml \\
 \hline
\end{tabular}
\caption{Electrolyte Composition}
\end{center}
\end {table}


One of the basic component of the liquid electrolyte is the organic solvent. It helps us in giving an environment for the iodide and the triiodide ions and helps in the dissolution and diffusion. The physical characteristics of the electrolyte further include viscosity, donor number, dielectric constant etc., this affects the photovoltaic performance of the dye sensitized solar cell. This solution should further have long term stability along with chemical, thermal, electrochemical stability while maintaining the need to avoid degradation and desorption of the dye from the oxide layer. However we need to make sure that it does not exhibit a significant absorption in the visible light range.
\begin{figure}[h]
\centering
\includegraphics[width = 100mm]{51.jpg}
\end{figure}
\newpage
\section{Assembly}

\begin{figure}[h]
\centering
\includegraphics[width = 50mm]{52.jpg}
\includegraphics[width = 65mm]{53.jpg}
\end{figure}
Stirring the electrolyte using magnetic stirrer. Electrolyte is injected between the electrodes using syringe. Clips are used to assemble electrodes. Platinum electrode is connected with positive supply. Between the electrodes electrolyte is used carefully. Artificial sunlight at 100 mW/cm2 with 31 degree Celsius is used. Temperature needs to be maintained for proper result.
\begin{figure}[h]
\centering
\includegraphics[width = 140mm]{54.jpg}
\end{figure}