Dye Sensitized solar cells or popularly known as DSSC, are thin film solar cell. It is essentially a semiconductor formed between an Electrolyte and Photo sensitized anode. In modern terminology, it is also known as Gratzel cell. This new class of advanced solar cell can be likened to artificial photosynthesis due to the way in which it mimics nature’s absorption of light energy. Currently, its conversion efficiency is between 8 to
and 11 percent(limited by the problems associated with both the electrolyte solution and the dye used), which is lower than most of other current solar technology. However, DSSC is a revolutionary technology that can be used to produce electricity in a wide variety of light conditions, indoors and outdoors, enabling the user to convert both artificial and natural light into energy to power a broad range of devices.

In this report, after getting an introduction to the device and its working properties we aim to study and demonstrate the broad range of industrial processes underlying the device. These processes are known as Electrophoretic planar deposition method and will be one of the important aspect of our work. There are variety of methods involved, and the aforementioned device will be made as per the given guidelines. Our remaining project will aim to synthesize multiple such prototype devices and measure and observe the various electrical and optical properties between them.



