\begin{center}
{\psfig{figure= Figures/1.jpg ,width=70mm }\\}
\end{center}
The experiment started with an on hand study on solar devices and how technology has evolved into dye sensitized solar cell. Further study was done to learn about the structure and fabrication techniques used. Electrophoretic deposition was studied ahead and how the electrolytic process helps in the various parameters involved in solar devices. We then moved to laboratory work where various synthesization components were studied and worked under. All the various processes including cleaning, layer deposition, electrolyte preparation was done according to the proper procedure. A multitude of devices were made for observation and correct comparison. The reading observed after proper calculation however were substandard and below the normal expected values. This was attributed to various factors which are mentioned in the report. In the closing, it can said that the device produced was successful in implementation and can be termed satisfactory.
\section{Scope of further work}
A lot of study and research has been carried out on DSSC. Different aspects of Dye sensitized solar cell have been focused and investigations have been thus carried on. Many sensitizers including inorganic and organic dyes have been used. The major challenge in the fabrication and commercialization of DSSCs is the low conversion efficiency and stability of the cell. The degradation of the cell based on dye sensitization, undesirable electrolyte properties and poor contact with the electrodes are the main causes of the poor performance of DSSCs. To enhance the performance of the DSSCs, several research directions are suggested: 
•	improving the dye stability by finding the optimum parameters to slow the dye degradation; 
•	 improving the dye structure to absorb more light at longer wavelengths, 780-2500 nm (the near infrared region, NIR); 
•	 improving the morphology of semiconductors to attaining the best electronic conduction to reduce the dark current;
•	Using dye and electrolyte additives to enhance the cell performance; and
•	Improving the mechanical contact between the two electrode 
Thus, the choice of materials is very important in the fabrication and deployment of DSSCs because the conversion efficiency and stability of the cell do not depend on a single factor alone

